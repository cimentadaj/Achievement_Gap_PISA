\documentclass[11pt, a4paper]{article}\usepackage[]{graphicx}\usepackage[]{color}
%% maxwidth is the original width if it is less than linewidth
%% otherwise use linewidth (to make sure the graphics do not exceed the margin)
\makeatletter
\def\maxwidth{ %
  \ifdim\Gin@nat@width>\linewidth
    \linewidth
  \else
    \Gin@nat@width
  \fi
}
\makeatother

\definecolor{fgcolor}{rgb}{0.345, 0.345, 0.345}
\newcommand{\hlnum}[1]{\textcolor[rgb]{0.686,0.059,0.569}{#1}}%
\newcommand{\hlstr}[1]{\textcolor[rgb]{0.192,0.494,0.8}{#1}}%
\newcommand{\hlcom}[1]{\textcolor[rgb]{0.678,0.584,0.686}{\textit{#1}}}%
\newcommand{\hlopt}[1]{\textcolor[rgb]{0,0,0}{#1}}%
\newcommand{\hlstd}[1]{\textcolor[rgb]{0.345,0.345,0.345}{#1}}%
\newcommand{\hlkwa}[1]{\textcolor[rgb]{0.161,0.373,0.58}{\textbf{#1}}}%
\newcommand{\hlkwb}[1]{\textcolor[rgb]{0.69,0.353,0.396}{#1}}%
\newcommand{\hlkwc}[1]{\textcolor[rgb]{0.333,0.667,0.333}{#1}}%
\newcommand{\hlkwd}[1]{\textcolor[rgb]{0.737,0.353,0.396}{\textbf{#1}}}%
\let\hlipl\hlkwb

\usepackage{framed}
\makeatletter
\newenvironment{kframe}{%
 \def\at@end@of@kframe{}%
 \ifinner\ifhmode%
  \def\at@end@of@kframe{\end{minipage}}%
  \begin{minipage}{\columnwidth}%
 \fi\fi%
 \def\FrameCommand##1{\hskip\@totalleftmargin \hskip-\fboxsep
 \colorbox{shadecolor}{##1}\hskip-\fboxsep
     % There is no \\@totalrightmargin, so:
     \hskip-\linewidth \hskip-\@totalleftmargin \hskip\columnwidth}%
 \MakeFramed {\advance\hsize-\width
   \@totalleftmargin\z@ \linewidth\hsize
   \@setminipage}}%
 {\par\unskip\endMakeFramed%
 \at@end@of@kframe}
\makeatother

\definecolor{shadecolor}{rgb}{.97, .97, .97}
\definecolor{messagecolor}{rgb}{0, 0, 0}
\definecolor{warningcolor}{rgb}{1, 0, 1}
\definecolor{errorcolor}{rgb}{1, 0, 0}
\newenvironment{knitrout}{}{} % an empty environment to be redefined in TeX

\usepackage{alltt}
\bibliographystyle{apalike}
\pagestyle{headings}

\usepackage{pdflscape}
\usepackage{subcaption}
\usepackage{graphics}
\usepackage{graphicx}
\usepackage[round, colon]{natbib}
\usepackage[colorlinks]{hyperref}
\AtBeginDocument{%
  \hypersetup{
    citecolor=blue,
    linkcolor=blue,   
    urlcolor=blue}}

\title{First paper - Draft}
\author{Jorge Cimentada}
\IfFileExists{upquote.sty}{\usepackage{upquote}}{}
\begin{document}
\setlength{\parindent}{2em}
\setlength{\parskip}{1em}
\showboxdepth=5
\showboxbreadth=5

\maketitle





\tableofcontents

\section{Analysis}









% This chunk needs to have cache = TRUE once you want to run it for like 2 hours until 
% the results_* models have been created once and the cache can save them



% Figure out how to ouput data frames as latex tables.






{\centering \includegraphics[width=5.5in,height=5in]{figure/correlation_incomeineq-1} 

}






{\centering \includegraphics[width=5.5in,height=5in]{figure/graphing_9010gaps-1} 

}






{\centering \includegraphics[width=5.5in,height=5in]{figure/graphing_allgaps-1} 

}






{\centering \includegraphics[width=5.5in,height=5in]{figure/graphing_ses_growth-1} 

}




% How to output data frames are latex tables?




{\centering \includegraphics[width=5.5in,height=5in]{figure/rate_change_graph-1} 

}




% How to output data frames are latex tables?





\section(Literature Review)

Recent research on educational inequality has found that differences in test peformance between High-SES and Low-SES kids has been growing quickly over the years. The literature on educational inequality has mainly concentrated on the United States (reardon here) but other international evidence is emerging with other similar findings for other countries. Despite this, the foremost and most important study on the subject is Reardon (2011) with American data. This is the cases, firstly, because it is the only country where cognitive testing is very widespread across surveys. And secondly, this trend has allowed to have testing records as early as 1940 (check reardon) until present day. Using this information, Reardon (2011) is the first to investigate the evolution of the cognitive gap and the results are very surprising. Not only has the cognitive gap between the 90th income percentile and the 10th income percentile grown over time, but it has grown faster and to be wider than the highly contested white-black gap (cite study of white-black gap). The widening of the achievement gap has been happening in parallel to the growth of income inequality. Although very suggestive, it is hard to link both things causally.

Many authors have taken this analysis to an international context in order to discover between-country trends. The comprehensive work of Corak and Waldfogel () shows studies, Australia, United Kingdom, United States and Canada. Their research design is very distinctive in that they use longitudinal data from children as early as age 2 and study the evolution of the achievement gap up until age 14 \footnote{To the best of my knowledge this is the only study that uses panel data to study achievement gaps, let alone to do this between country}. The core finding behind the book is that the American achievement gap is much wider than in any other country. More specifically, the American achievement gap has been seen to be around ~ 1.25/1.50 standard deviations and doesn't seem to change much across the lifetime. That is, they find that the achievement gap is very stable all the way from early education to secondary school (although their studies cover up until age 14). Despite their rigurosity, the four surveys have significant differences and cannot be readily compared. This makes their results more suggestive rather than definitive. Which is why authors such as Chmielewski and Reardon (2016) and Chmielewski (2017) have attempted not only to compare gaps between countries, but to evaluate whether there is a general increase in educational inequality in many countries.  Chmielewski and Reardon (2016), using the Programme for International Student Assessment (PISA), the Trends in International Mathematics and Science Study (TIMSS) and PIRLS (the Progress in International Reading Literacy Study), the assess whether there are differences in the achievement gap between countries. Although there is potential to study countries across years using these studies, given their researchn design it's impossible and only 3 countries have enough data points to be studied longitudinally. None the less, they find that there is considerable variation in the achievement gap between top and bottom earning families across many developed countries. They go even further and link this achievement gap to several country-level indicators related to income inequality, school differentiation, central exams, etc ... The correlations are indeed very suggestive but clearly we must be extremely cautios in drawing causal changes from these two variables. But it is important to stress that their design is quite different from the work of Reardon (2011). Reardon (2011) takes studies in the U.S starting from the 1940's until 2015 \footnote{Each study is independent of each other meaning that the it might've been given to 6 year olds as well as to 12 year olds. Although the author adjusts for age, the trends can't be generalized to age-specific, but rather in overall terms.} and makes them comparable across time. This gives the author a very long time series to build a reliable achievement gap (over 40 years). In their study, Chmielewski and Reardon (2016) change the aim of the study to model between-country differences from a cross-sectional perspective.

Chmielewski (2017), building on the work of Chmielewski and Reardon (2016) and Reardon and Portilla (2015) pooled together all the previously mentioned data, together with over 10 more studies ranging from the year 1964 until 2015 in order to discover differences between and across countries. With over 50 years of data, and over 100 countries, Chmielewski (2017) finds that there seems to be a general world increase in the achievement gap. However, once she disentangles the relationship by country, she finds a reasonable amount of heterogeneity, with some countries seeing the achievement gap closing, others no change at all, while others record a steady increase. One clear limitation of their study (as well as Reardon (2011)) is that the adjust for the age of each child in all studies. Although for their purposes is the right thing to do \footnote{The differences in achievement could simply be due to changes in cognitive abilities across the lifetime. However, as we've noted before, Corak and Waldfogel (2015) find that the achievement gap is very stable across the life time}, they are masking age-specific achievement gaps by controlling for age, such as Reardon (2011) did.

The evolution of High/Low SES gaps for preschool children might be much less marked than the same gap for high school children. The explanation, although very debated, has been gaining much support in recent years. In countries with high levels of curricular differentiation, the transition from early schooling into the tracking system has been found to increase inequality of learning (Wossman and Hanushek diff-in-diff). Moreover, the vast sociological literature on educational transitions systematically finds that tracking tends to foster between-track inequality rather than erode their differences by tackling their specific needs (Van der Werfhorst and Mijs 2011). Based on this, we cannot simply assume that the achievement gap has been neither constant across cohorts (because there have been tracking reforms in many countries, introducing as well as elimination tracking structures) nor the same between ages, because tracking/no tracking might exarcebate the achievement gap.

With this being said, this paper introduces one novelty in the literature which is to evaluate the evolution of the High/Low SES achievement gap in the past 15 years for all PISA participant countries. This is different from previous work because it concentrates solely on 15 year old children, and it attempts to capture the evolution of the achievement gap for each country. The advantages of this study are numerous. First, we concentrate on the evolution of the gap for only 15 year olds. This is different from all the work of Chmielewski and Reardon, in which they pool all ages to get average changes, to study the properties of specific age-groups. As we've seen before, there are reasons to think that specific age-groups have seen changes in the achievement gap. Moreover, in almost all countries with a tracked curriculum children are either at or in the process of tracking by the age 15, meaning that we will be able to link whether tracked countries are the most variable in their evolution of achievement gaps.

Talk about reardon and portilla and how the achievement gap has been closing in the last 15 years.
Talk about how the achievement gap has been widening in Malaysia and South Korea (or Japan?)
Note how many of these studies haven't really concentrated on who is getting better or worse: top or bottom?
Talk more about how countries with high social mobility has been linked to smallest achievement gaps

\section(Research design)

\end{document}
