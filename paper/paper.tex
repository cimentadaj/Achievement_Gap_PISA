\documentclass[11pt, a4paper]{article}\usepackage[]{graphicx}\usepackage[]{color}
%% maxwidth is the original width if it is less than linewidth
%% otherwise use linewidth (to make sure the graphics do not exceed the margin)
\makeatletter
\def\maxwidth{ %
  \ifdim\Gin@nat@width>\linewidth
    \linewidth
  \else
    \Gin@nat@width
  \fi
}
\makeatother

\definecolor{fgcolor}{rgb}{0.345, 0.345, 0.345}
\newcommand{\hlnum}[1]{\textcolor[rgb]{0.686,0.059,0.569}{#1}}%
\newcommand{\hlstr}[1]{\textcolor[rgb]{0.192,0.494,0.8}{#1}}%
\newcommand{\hlcom}[1]{\textcolor[rgb]{0.678,0.584,0.686}{\textit{#1}}}%
\newcommand{\hlopt}[1]{\textcolor[rgb]{0,0,0}{#1}}%
\newcommand{\hlstd}[1]{\textcolor[rgb]{0.345,0.345,0.345}{#1}}%
\newcommand{\hlkwa}[1]{\textcolor[rgb]{0.161,0.373,0.58}{\textbf{#1}}}%
\newcommand{\hlkwb}[1]{\textcolor[rgb]{0.69,0.353,0.396}{#1}}%
\newcommand{\hlkwc}[1]{\textcolor[rgb]{0.333,0.667,0.333}{#1}}%
\newcommand{\hlkwd}[1]{\textcolor[rgb]{0.737,0.353,0.396}{\textbf{#1}}}%
\let\hlipl\hlkwb

\usepackage{framed}
\makeatletter
\newenvironment{kframe}{%
 \def\at@end@of@kframe{}%
 \ifinner\ifhmode%
  \def\at@end@of@kframe{\end{minipage}}%
  \begin{minipage}{\columnwidth}%
 \fi\fi%
 \def\FrameCommand##1{\hskip\@totalleftmargin \hskip-\fboxsep
 \colorbox{shadecolor}{##1}\hskip-\fboxsep
     % There is no \\@totalrightmargin, so:
     \hskip-\linewidth \hskip-\@totalleftmargin \hskip\columnwidth}%
 \MakeFramed {\advance\hsize-\width
   \@totalleftmargin\z@ \linewidth\hsize
   \@setminipage}}%
 {\par\unskip\endMakeFramed%
 \at@end@of@kframe}
\makeatother

\definecolor{shadecolor}{rgb}{.97, .97, .97}
\definecolor{messagecolor}{rgb}{0, 0, 0}
\definecolor{warningcolor}{rgb}{1, 0, 1}
\definecolor{errorcolor}{rgb}{1, 0, 0}
\newenvironment{knitrout}{}{} % an empty environment to be redefined in TeX

\usepackage{alltt}
\bibliographystyle{apalike}
\pagestyle{headings}

\usepackage{pdflscape}
\usepackage{subcaption}
\usepackage{graphics}
\usepackage{graphicx}
\usepackage[round, colon]{natbib}
\usepackage[colorlinks]{hyperref}
\AtBeginDocument{%
  \hypersetup{
    citecolor=blue,
    linkcolor=blue,   
    urlcolor=blue}}

\title{First paper - Draft}
\author{Jorge Cimentada}
\IfFileExists{upquote.sty}{\usepackage{upquote}}{}
\begin{document}
\setlength{\parindent}{2em}
\setlength{\parskip}{1em}
\showboxdepth=5
\showboxbreadth=5

\maketitle





\tableofcontents

\section{Analysis}















{\centering \includegraphics[width=5.5in,height=5in]{figure/graphin_differences-1} 

}








{\centering \includegraphics[width=5.5in,height=5in]{figure/unnamed-chunk-2-1} 

}






Recent research on educational inequality has found that differences in test peformance between High-SES and Low-SES kids has been growing quickly over the years. The vast literature on educational inequality has mainly concentrated on the United States (reardon here). First, because it is the only country where cognitive testing is very widespread across surveys. And secondly, this trend has allowed to have testing records as early as 1940 (check reardon) until present day. Using this information, Reardon (2011) is the first to investigate the evolution of the cognitive gap and the results are very surprising. Not only has the cognitive gap between the 90th income percentile and the 10th income percentile grown over time, but it has grown faster and to be wider than the highly contested white-black gap (cite study of white-black gap).

The widening of the achievement gap has been happening in parallel to the growth of income inequality. Although very suggestive, it is hard to link both things causally. The downside of this research study is that it is mainly US-based, and not a lot of evidence is available in an international context. To the best of my knowledge, the only attempt to take this to an international context is the work by Chmielewski and Reardon (2016). They find that there is considerable variation in the achievement gap between top and bottom earning families. However, their design is quite different from the one from Reardon (2011). Reardon (2011) takes studies in the U.S starting from the 1940's until 2015 \footnote{Each study is independent of each other meaning that the it might've been given to 6 year olds as well as to 12 year olds. Although the author adjusts for age, the trends can't be generalized to age-specific, but rather in overall terms.} and makes them comparable across time. That is, he builds a trend of achievement gap. Chmielewski and Reardon (2016) use the international surveys PISA and PIRLS to compare the achievement gap across countries. Unfortunately, very few countries can be compared across time in PISA.

This paper introduces one novelty in the literature which is to evaluate the evolution of the High/Low SES achievement gap in the past 15 years for all PISA participant countries. This is different from previous work because it concentrates solely on 15 year old children, and it attempts to capture the evolution of the achievement gap for each country. 

Do we need to distinguish between age-group achievement gaps?

\end{document}
